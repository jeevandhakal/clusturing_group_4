% ==== Appendix A Tables ==== %
% LaTeX will automatically label this as "Appendix A."
\section[Appendix]{Appendix}
\label{sec:app_a}

\subsection{Source Code - GitHub}
\href{https://github.com/jeevandhakal/clusturing_group_4}{GitHub repository}

\subsection{YouTube Video}
\href{https://youtu.be/r8s4AN-6B1M}{YouTube Video}

\subsection{Chat Session Links}
\href{https://notebooklm.google.com/notebook/d60960bd-4ada-4bbe-b8cf-2f64da0a0d99?authuser=1}{Notebook LM}

\href{https://chat.deepseek.com/a/chat/s/6e1b2b7c-6afb-4fd1-9d7d-a9f8f569cb7d}{DeepSeek 1}

\href{https://chat.deepseek.com/a/chat/s/c3b5cd3c-7d4b-473d-b1ab-1353823a14d1}{DeepSeek 2}


\subsection {Additional Plots}
\begin{figure}[H]
    \centering
    \includegraphics[width=0.9\textwidth]{figures/p_clusters_scatter_plot.png}
    \caption{Dual-metric analysis comparing the elbow method and silhouette score}
    \label{fig:product_elbow_silhouette}
\end{figure}


\begin{figure}[H]
    \centering
    \includegraphics[width=0.9\textwidth]{figures/p_elbow.png}
    \caption{Product Clusters Analysis}
    \label{fig:product_cluster}
\end{figure}


\begin{figure}[H]
    \centering
    \includegraphics[width=0.9\textwidth]{figures/cx_k_3_1.png}
    \caption{Various plots for features for 3 clusters}
    \label{fig:cx_k_3_1}
\end{figure}


\begin{figure}[H]
    \centering
    \includegraphics[width=0.9\textwidth]{figures/cx_k_3_2.png}
    \caption{Various plots for features for 3 clusters}
    \label{fig:cx_k_3_2}
\end{figure}

\begin{figure}[H]
    \centering
    \includegraphics[width=0.9\textwidth]{figures/cx_k_4_1.png}
    \caption{Various plots for features for 4 clusters}
    \label{fig:cx_k_4_1}
\end{figure}

\begin{figure}[H]
    \centering
    \includegraphics[width=0.9\textwidth]{figures/cx_k_3_4.png}
    \caption{Comparison between 3 and 4 clusters}
    \label{fig:cx_k_3_4}
\end{figure}

